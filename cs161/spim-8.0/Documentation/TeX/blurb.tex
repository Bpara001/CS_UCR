% Master File: blurb.tex
% Document Type: LaTeX

\documentstyle[11pt,regular-margins]{article}
\begin{document}

\begin{center}
  \LARGE\bf SPIM: A MIPS R2000 Simulator
\end{center}

The SPIM S20 is a software simulator that executes programs for the
MIPS R2000/R3000 RISC computers.  SPIM can read and immediately
execute files containing assembly language statements or MIPS-produced
binary a.out files.  SPIM is a self-contained system for running these
programs and contains a debugger and interface to the operating
system.

I wrote SPIM as the target machine for an undergraduate compiler
course. SPIM is very portable (it has been tested on DECStations, Sun
3s, Sun 4s, PC/RTs, and Sequent Symmetries), so students are able to
generate code for a simple, clean, orthogonal computer---no matter
which god-awful machine they used.  It was a very successful in this
role and has been used in several other compiler courses.  It has also
been modified and used for computer architecture studies.

SPIM is fairly slow.  It runs about 1000 dhrystones/second, which is
roughly 1/25th the speed of a DECStation 3100, or about the speed of a
68010-based system.

SPIM implements almost the entire MIPS assembler-extended instruction
set (I've omitted some the complex floating point comparisons and
details of the memory system page tables).  SPIM comes with complete
source code and documentation of all instructions (including several
that aren't in Kane's book, but are produced by MIPS compilers).  It
also include a large torture test to verify a port to a new machine.

SPIM has a simple, terminal-style and a flashy, X-windows interface.

SPIM is copyrighted by me and distributed under a BSD License.

\begin{quote}
James Larus\\
Computer Sciences Department\\
1210 West Dayton Street\\
University of Wisconsin\\
Madison, WI 53706 USA\\
(608) 262-9519
\end{quote}
\end{document}
