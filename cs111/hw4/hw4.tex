
\documentclass{article}


\usepackage{fullpage,latexsym,picinpar,amsmath,amsfonts,graphicx}

\input{./macros.tex}

\begin{document}

\centerline{\large \bf CS/MATH111 ASSIGNMENT 4}
\centerline{due Tuesday, May~24, midnight}

\vskip 0.2in
\noindent{\bf Individual assignment:} Problems 1 and 2.

\noindent{\bf Group assignment:} Problems 1,2 and 3.

\vskip 0.1in

%%%%%%%%%%%%%%%%%%%%%%%%%%%%

\begin{problem}
Give the asymptotic value (using the $\Theta$-notation)
for the number of words that will be printed by the algorithms below.
Your solution needs to consist of an appropriate recurrence 
equation and its solution, with a brief justification.
(See the suggested format at the bottom of the assignment).

\bigskip
\noindent
(a)\ \ 
\begin{minipage}[t]{3in}
\begin{tabbing}
aaa \= aaa \= aaa \= aaa \=  \kill
\textbf{Algorithm} \textsc{Jazz} $(n: \mbox{\bf integer})$ \\
          \> \textbf{if} $n = 1$ \\
          \>\>  print(``jazz") \\
          \>\textbf{else} \\
      		\>\> \textbf{for} $i \leftarrow 1$ \textbf{to} $2n$ \textbf{do} print(``jazz")\\
          \>\>  \textbf{for} $j\leftarrow 1$ \textbf{to} $4$ 
				\textbf{do} \textsc{Jazz}$(\ceiling{n/4})$
\end{tabbing}
\end{minipage}

\bigskip
\noindent
(b)\ \
\begin{minipage}[t]{3in}
\begin{tabbing}
aaa \= aaa \= aaa \= aaa \=  \kill
\textbf{Algorithm} \textsc{Salsa} $(n: \mbox{\bf integer})$ \\
          \> \textbf{if} $n = 1$ \\
          \>\>  print(``salsa") \\
          \>\textbf{else} \\
          \>\>  \textbf{for} $j \leftarrow 1$ \textbf{to} $8$ 
					\textbf{do} \textsc{Salsa}$(\floor{n/2})$\\
      \>\> \textbf{for} $i \leftarrow 1$ \textbf{to} $n^2$ \textbf{do} print(``salsa")
\end{tabbing}
\end{minipage}

\bigskip
\noindent
(c)\ \ 
\begin{minipage}[t]{3in}
\begin{tabbing}
aaa \= aaa \= aaa \= aaa \=  \kill
\textbf{Algorithm} \textsc{Mariachi} $(n: \mbox{\bf integer})$ \\
          \> \textbf{if} $n = 1$ \\
          \>\>  print(``mariachi") \\
          \>\textbf{else} \\
          \>\>  \textsc{Mariachi}$(\ceiling{n/4})$\\
          \>\>  \textsc{Mariachi}$(\ceiling{n/4})$\\
          \>\>  \textsc{Mariachi}$(\ceiling{n/4})$\\
      \>\> \textbf{for} $i \leftarrow 1$ \textbf{to} $3n$ \textbf{do} print(``mariachi")
\end{tabbing}
\end{minipage}

\bigskip
\noindent
(d)\ \ 
\begin{minipage}[t]{3in}
\begin{tabbing}
aaa \= aaa \= aaa \= aaa \=  \kill
\textbf{Algorithm} \textsc{Reggae} $(n: \mbox{\bf integer})$ \\
          \> \textbf{if} $n = 1$ \\
          \>\>  print(``reggae") \\
          \>\textbf{else} \\
          \>\>  \textsc{Reggae}$(\ceiling{n/3})$\\
          \>\>  \textsc{Reggae}$(\floor{n/3})$\\
      \>\> \textbf{for} $i \leftarrow 1$ \textbf{to} $7$ \textbf{do} print(``reggae")
\end{tabbing}
\end{minipage}
\bigskip
\ 


\bigskip
\noindent
(e)\ \ 
\begin{minipage}[t]{3in}
\begin{tabbing}
aaa \= aaa \= aaa \= aaa \=  \kill
\textbf{Algorithm} \textsc{Rhumba} $(n: \mbox{\bf integer})$ \\
          \> \textbf{if} $n = 1$ \\
          \>\>  print(``rhumba") \\
          \>\textbf{else} \\
          \>\>  \textbf{for} $j \leftarrow 1$ \textbf{to} $16$ 
					\textbf{do} \textsc{Rhumba}$(\floor{n/4})$\\
      \>\> \textbf{for} $i \leftarrow 1$ \textbf{to} $2n^3$ \textbf{do} print(``rhumba")
\end{tabbing}
\end{minipage}
\bigskip
\


\end{problem}


%%%%%%%%%%%%%%%%%%%%%%%%%%%%

\begin{problem}
Determine (using the inclusion-exclusion principle)
the number of integer solutions of the equation:
%
\begin{eqnarray*}
x + y + z &=& 17,
\end{eqnarray*}
%
under the constraints 
%
\begin{eqnarray*}
        0 \;\le\; x \;\le\; 4 \\
        0 \;\le\; y \;\le\; 6 \\
        0\;\le\; z  \;\le\; 9
\end{eqnarray*}
%
Show your work.
\end{problem}

%%%%%%%%%%%%%%%%%%%%%%%%%%%%

\begin{problem}
Suppose we have three sets, $A$, $B$, $C$
with the following properties:

\begin{description}

\item{(a)} $|A| = 75$, $|B| = 94$,
                $|C| = 69$,

\item{(b)} $|A\cap B| = 61$,
        $|A\cap C| = 37$,
        $|B\cap C| = 50$,

\item{(c)}
$|A\cup B\cup C| = 4\cdot|A\cap B\cap C|$.

\end{description}

Use the inclusion-exclusion formula to
determine the number of elements in $A\cup B\cup C$.
Show your work.
(Hint: Use the inclusion-exclusion principle to
set up an appropriate equation, and solve it.)
\end{problem}

%%%%%%%%%%%%%%%%%%%%%%%%%%%%

\vskip 0.1in
\paragraph{Submission.}
You need to turn in your homework via Ilearn, by midnight,
Tuesday, May 24. 

\vskip 0.5in

\noindent
\hrule


\vskip 0.3in

\paragraph{Suggested format for solutions for Problem 1:}
 ... There are $7$ recursive calls, each with parameter $\ceiling{n/2}$. Since
we are looking for an asymptotic solution, we can ignore rounding. Then
the number of letters printed can be expressed by the recurrence:
%
\begin{eqnarray*}
	X(n) &=& 7 X(n/2) + 4n^2.
\end{eqnarray*}
%
We apply the Master Theorem with $a = 7$, $b = 2$, $d = 2$. Here, we
have $a > b^d$, so the solution is $\Theta(n^{\log 7})$.

\end{document}

