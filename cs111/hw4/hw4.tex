
\documentclass{article}


\usepackage{fullpage,latexsym,picinpar,amsmath,amsfonts,graphicx}

           

%%%%%%%%%%%%%%%%%%%%%%%%%%%%%%%%%%%%%%%%%%%%%%%%%%%%%%%%%%%%%%%%%%%%%%%%%%%%%%%%%%%
%%%%%%%%%%%  LETTERS 
%%%%%%%%%%%%%%%%%%%%%%%%%%%%%%%%%%%%%%%%%%%%%%%%%%%%%%%%%%%%%%%%%%%%%%%%%%%%%%%%%%%

\newcommand{\barx}{{\bar x}}
\newcommand{\bary}{{\bar y}}
\newcommand{\barz}{{\bar z}}
\newcommand{\bart}{{\bar t}}

\newcommand{\bfP}{{\bf{P}}}

%%%%%%%%%%%%%%%%%%%%%%%%%%%%%%%%%%%%%%%%%%%%%%%%%%%%%%%%%%%%%%%%%%%%%%%%%%%%%%%%%%%
%%%%%%%%%%%%%%%%%%%%%%%%%%%%%%%%%%%%%%%%%%%%%%%%%%%%%%%%%%%%%%%%%%%%%%%%%%%%%%%%%%%
                                                                                
\newcommand{\parend}[1]{{\left( #1  \right) }}
\newcommand{\spparend}[1]{{\left(\, #1  \,\right) }}
\newcommand{\angled}[1]{{\left\langle #1  \right\rangle }}
\newcommand{\brackd}[1]{{\left[ #1  \right] }}
\newcommand{\spbrackd}[1]{{\left[\, #1  \,\right] }}
\newcommand{\braced}[1]{{\left\{ #1  \right\} }}
\newcommand{\leftbraced}[1]{{\left\{ #1  \right. }}
\newcommand{\floor}[1]{{\left\lfloor #1\right\rfloor}}
\newcommand{\ceiling}[1]{{\left\lceil #1\right\rceil}}
\newcommand{\barred}[1]{{\left|#1\right|}}
\newcommand{\doublebarred}[1]{{\left|\left|#1\right|\right|}}
\newcommand{\spaced}[1]{{\, #1\, }}
\newcommand{\suchthat}{{\spaced{|}}}
\newcommand{\numof}{{\sharp}}
\newcommand{\assign}{{\,\leftarrow\,}}
\newcommand{\myaccept}{{\mbox{\tiny accept}}}
\newcommand{\myreject}{{\mbox{\tiny reject}}}
\newcommand{\blanksymbol}{{\sqcup}}
                                                                                                                         
\newcommand{\veps}{{\varepsilon}}
\newcommand{\Sigmastar}{{\Sigma^\ast}}
                           
\newcommand{\half}{\mbox{$\frac{1}{2}$}}    
\newcommand{\threehalfs}{\mbox{$\frac{3}{2}$}}   
\newcommand{\domino}[2]{\left[\frac{#1}{#2}\right]}  

%%%%%%%%%%%% complexity classes

\newcommand{\PP}{\mathbb{P}}
\newcommand{\NP}{\mathbb{NP}}
\newcommand{\PSPACE}{\mathbb{PSPACE}}
\newcommand{\coNP}{\textrm{co}\mathbb{NP}}
\newcommand{\DLOG}{\mathbb{L}}
\newcommand{\NLOG}{\mathbb{NL}}
\newcommand{\NL}{\mathbb{NL}}

%%%%%%%%%%% decision problems

\newcommand{\PCP}{\sc{PCP}}
\newcommand{\Path}{\sc{Path}}
\newcommand{\GenGeo}{\sc{Generalized Geography}}

\newcommand{\malytm}{{\mbox{\tiny TM}}}
\newcommand{\malycfg}{{\mbox{\tiny CFG}}}
\newcommand{\Atm}{\mbox{\rm A}_\malytm}
\newcommand{\complAtm}{{\overline{\mbox{\rm A}}}_\malytm}
\newcommand{\AllCFG}{{\mbox{\sc All}}_\malycfg}
\newcommand{\complAllCFG}{{\overline{\mbox{\sc All}}}_\malycfg}
\newcommand{\complL}{{\bar L}}
\newcommand{\TQBF}{\mbox{\sc TQBF}}
\newcommand{\SAT}{\mbox{\sc SAT}}

%%%%%%%%%%%%%%%%%%%%%%%%%%%%%%%%%%%%%%%%%%%%%%%%%%%%%%%%%%%%%%%%%%%%%%%%%%%%%%%%%%%
%%%%%%%%%%%%%%% for homeworks
%%%%%%%%%%%%%%%%%%%%%%%%%%%%%%%%%%%%%%%%%%%%%%%%%%%%%%%%%%%%%%%%%%%%%%%%%%%%%%%%%%%

\newcommand{\student}[2]{%
{\noindent\Large{ \emph{#1} SID {#2} } \hfill} \vskip 0.1in}

\newcommand{\assignment}[1]{\medskip\centerline{\large\bf CS 111 ASSIGNMENT {#1}}}

\newcommand{\duedate}[1]{{\centerline{due {#1}\medskip}}}     

\newcounter{problemnumber}                                                                                 

\newenvironment{problem}{{\vskip 0.1in \noindent
              \bf Problem~\addtocounter{problemnumber}{1}\arabic{problemnumber}:}}{}

\newcounter{solutionnumber}

\newenvironment{solution}{{\vskip 0.1in \noindent
             \bf Solution~\addtocounter{solutionnumber}{1}\arabic{solutionnumber}:}}
				{\ \newline\smallskip\lineacross\smallskip}

\newcommand{\lineacross}{\noindent\mbox{}\hrulefill\mbox{}}

\newcommand{\decproblem}[3]{%
\medskip
\noindent
\begin{list}{\hfill}{\setlength{\labelsep}{0in}
                       \setlength{\topsep}{0in}
                       \setlength{\partopsep}{0in}
                       \setlength{\leftmargin}{0in}
                       \setlength{\listparindent}{0in}
                       \setlength{\labelwidth}{0.5in}
                       \setlength{\itemindent}{0in}
                       \setlength{\itemsep}{0in}
                     }
\item{{{\sc{#1}}:}}
                \begin{list}{\hfill}{\setlength{\labelsep}{0.1in}
                       \setlength{\topsep}{0in}
                       \setlength{\partopsep}{0in}
                       \setlength{\leftmargin}{0.5in}
                       \setlength{\labelwidth}{0.5in}
                       \setlength{\listparindent}{0in}
                       \setlength{\itemindent}{0in}
                       \setlength{\itemsep}{0in}
                       }
                \item{{\em Instance:\ }}{#2}
                \item{{\em Query:\ }}{#3}
                \end{list}
\end{list}
\medskip
}

%%%%%%%%%%%%%%%%%%%%%%%%%%%%%%%%%%%%%%%%%%%%%%%%%%%%%%%%%%%%%%%%%%%%%%%%%%%%%%%%%%%
%%%%%%%%%%%%% for quizzes
%%%%%%%%%%%%%%%%%%%%%%%%%%%%%%%%%%%%%%%%%%%%%%%%%%%%%%%%%%%%%%%%%%%%%%%%%%%%%%%%%%%

\newcommand{\quizheader}{ {\large NAME: \hskip 3in SID:\hfill}
                                \newline\lineacross \medskip }


%%%%%%%%%%%%%%%%%%%%%%%%%%%%%%%%%%%%%%%%%%%%%%%%%%%%%%%%%%%%%%%%%%%%%%%%%%%%%%%%%%%
%%%%%%%%%%%%% for final
%%%%%%%%%%%%%%%%%%%%%%%%%%%%%%%%%%%%%%%%%%%%%%%%%%%%%%%%%%%%%%%%%%%%%%%%%%%%%%%%%%%

\newcommand{\namespace}{\noindent{\Large NAME: \hfill SID:\hskip 1.5in\ }\\\medskip\noindent\mbox{}\hrulefill\mbox{}}



\begin{document}

\centerline{\large \bf CS/MATH111 ASSIGNMENT 4}
\centerline{due Tuesday, May~24, midnight}

\vskip 0.2in
\noindent{\bf Individual assignment:} Problems 1 and 2.

\noindent{\bf Group assignment:} Problems 1,2 and 3.

\vskip 0.1in

%%%%%%%%%%%%%%%%%%%%%%%%%%%%

\begin{problem}
Give the asymptotic value (using the $\Theta$-notation)
for the number of words that will be printed by the algorithms below.
Your solution needs to consist of an appropriate recurrence 
equation and its solution, with a brief justification.
(See the suggested format at the bottom of the assignment).

\bigskip
\noindent
(a)\ \ 
\begin{minipage}[t]{3in}
\begin{tabbing}
aaa \= aaa \= aaa \= aaa \=  \kill
\textbf{Algorithm} \textsc{Jazz} $(n: \mbox{\bf integer})$ \\
          \> \textbf{if} $n = 1$ \\
          \>\>  print(``jazz") \\
          \>\textbf{else} \\
      		\>\> \textbf{for} $i \leftarrow 1$ \textbf{to} $2n$ \textbf{do} print(``jazz")\\
          \>\>  \textbf{for} $j\leftarrow 1$ \textbf{to} $4$ 
				\textbf{do} \textsc{Jazz}$(\ceiling{n/4})$
\end{tabbing}
\end{minipage}

\bigskip
\noindent
(b)\ \
\begin{minipage}[t]{3in}
\begin{tabbing}
aaa \= aaa \= aaa \= aaa \=  \kill
\textbf{Algorithm} \textsc{Salsa} $(n: \mbox{\bf integer})$ \\
          \> \textbf{if} $n = 1$ \\
          \>\>  print(``salsa") \\
          \>\textbf{else} \\
          \>\>  \textbf{for} $j \leftarrow 1$ \textbf{to} $8$ 
					\textbf{do} \textsc{Salsa}$(\floor{n/2})$\\
      \>\> \textbf{for} $i \leftarrow 1$ \textbf{to} $n^2$ \textbf{do} print(``salsa")
\end{tabbing}
\end{minipage}

\bigskip
\noindent
(c)\ \ 
\begin{minipage}[t]{3in}
\begin{tabbing}
aaa \= aaa \= aaa \= aaa \=  \kill
\textbf{Algorithm} \textsc{Mariachi} $(n: \mbox{\bf integer})$ \\
          \> \textbf{if} $n = 1$ \\
          \>\>  print(``mariachi") \\
          \>\textbf{else} \\
          \>\>  \textsc{Mariachi}$(\ceiling{n/4})$\\
          \>\>  \textsc{Mariachi}$(\ceiling{n/4})$\\
          \>\>  \textsc{Mariachi}$(\ceiling{n/4})$\\
      \>\> \textbf{for} $i \leftarrow 1$ \textbf{to} $3n$ \textbf{do} print(``mariachi")
\end{tabbing}
\end{minipage}

\bigskip
\noindent
(d)\ \ 
\begin{minipage}[t]{3in}
\begin{tabbing}
aaa \= aaa \= aaa \= aaa \=  \kill
\textbf{Algorithm} \textsc{Reggae} $(n: \mbox{\bf integer})$ \\
          \> \textbf{if} $n = 1$ \\
          \>\>  print(``reggae") \\
          \>\textbf{else} \\
          \>\>  \textsc{Reggae}$(\ceiling{n/3})$\\
          \>\>  \textsc{Reggae}$(\floor{n/3})$\\
      \>\> \textbf{for} $i \leftarrow 1$ \textbf{to} $7$ \textbf{do} print(``reggae")
\end{tabbing}
\end{minipage}
\bigskip
\ 


\bigskip
\noindent
(e)\ \ 
\begin{minipage}[t]{3in}
\begin{tabbing}
aaa \= aaa \= aaa \= aaa \=  \kill
\textbf{Algorithm} \textsc{Rhumba} $(n: \mbox{\bf integer})$ \\
          \> \textbf{if} $n = 1$ \\
          \>\>  print(``rhumba") \\
          \>\textbf{else} \\
          \>\>  \textbf{for} $j \leftarrow 1$ \textbf{to} $16$ 
					\textbf{do} \textsc{Rhumba}$(\floor{n/4})$\\
      \>\> \textbf{for} $i \leftarrow 1$ \textbf{to} $2n^3$ \textbf{do} print(``rhumba")
\end{tabbing}
\end{minipage}
\bigskip
\


\end{problem}


%%%%%%%%%%%%%%%%%%%%%%%%%%%%

\begin{problem}
Determine (using the inclusion-exclusion principle)
the number of integer solutions of the equation:
%
\begin{eqnarray*}
x + y + z &=& 17,
\end{eqnarray*}
%
under the constraints 
%
\begin{eqnarray*}
        0 \;\le\; x \;\le\; 4 \\
        0 \;\le\; y \;\le\; 6 \\
        0\;\le\; z  \;\le\; 9
\end{eqnarray*}
%
Show your work.
\end{problem}

%%%%%%%%%%%%%%%%%%%%%%%%%%%%

\begin{problem}
Suppose we have three sets, $A$, $B$, $C$
with the following properties:

\begin{description}

\item{(a)} $|A| = 75$, $|B| = 94$,
                $|C| = 69$,

\item{(b)} $|A\cap B| = 61$,
        $|A\cap C| = 37$,
        $|B\cap C| = 50$,

\item{(c)}
$|A\cup B\cup C| = 4\cdot|A\cap B\cap C|$.

\end{description}

Use the inclusion-exclusion formula to
determine the number of elements in $A\cup B\cup C$.
Show your work.
(Hint: Use the inclusion-exclusion principle to
set up an appropriate equation, and solve it.)
\end{problem}

%%%%%%%%%%%%%%%%%%%%%%%%%%%%

\vskip 0.1in
\paragraph{Submission.}
You need to turn in your homework via Ilearn, by midnight,
Tuesday, May 24. 

\vskip 0.5in

\noindent
\hrule


\vskip 0.3in

\paragraph{Suggested format for solutions for Problem 1:}
 ... There are $7$ recursive calls, each with parameter $\ceiling{n/2}$. Since
we are looking for an asymptotic solution, we can ignore rounding. Then
the number of letters printed can be expressed by the recurrence:
%
\begin{eqnarray*}
	X(n) &=& 7 X(n/2) + 4n^2.
\end{eqnarray*}
%
We apply the Master Theorem with $a = 7$, $b = 2$, $d = 2$. Here, we
have $a > b^d$, so the solution is $\Theta(n^{\log 7})$.

\end{document}

