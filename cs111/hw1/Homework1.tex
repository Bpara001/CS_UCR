% <--- percent sign starts a comment line in LaTeX

%----------------------------------------------------------
% This is a sample assignment .tex file. Put your name,
% assignment number and the due date below, as shown.
% Before you typeset your own assignment try to preview
% and print this one as follows:
%   1. Save this in a file, say hw.tex
%   2. do "latex hw"
%	3. latex will produce a pdf file that should be named hw.pdf.
%	4. Use any pdf viewer to view the document
% ------------------------------------------------------------

\documentclass[11pt]{article}

\usepackage{fullpage,graphicx,latexsym,picinpar,amsbsy,amsmath,amsfonts}

\input{macros.tex}

\begin{document}

% v -- YOUR NAME and SID in the braces
\student{ Chris Wong}{ 860 923 521}
% v -- ASSIGNMENT NUMBER in the braces
\assignment{ 1 }
% v-- DUE DATE in the braces
\duedate{Tuesday, April 12 }

\medskip

%%%%%%%%%%%%%%%%%%%%%%%%%%%%%%%%%%%%%%%%%%%%%%%%%%%%%%%%%%%%%%%%%%%%%%%%%%

\lineacross

%%%%%%%%%%%%%%%%%%%%%%%%%%%%

\begin{problem}
	(a)
Give the exact formula (as a function of $n$) for the number of
times ``whatsup" is printed by Algorithm~\textsc{WhatsUp} below.
First express it as a summation, and then give a closed-form expression.

\noindent
(b)
Give the asymptotic value of the
number of ``whatsup"s (using the $\Theta$-notation.)

\begin{tabbing}
aa \= aa \= aa \= aa \= aa \= aa \= \kill
\textbf{Algorithm} \textsc{WhatsUp} $(n: \mbox{\bf integer})$ \\
      \> \textbf{for} $i \leftarrow 1$ \textbf{to} $2n$
                         \textbf{do} \\
      \> \> \textbf{for} $j \leftarrow 1$ \textbf{to} $(i+1)^2$ \textbf{do} \\
      \> \> \> print(``whatsup")
\end{tabbing}
\end{problem}

%---------------------------

\begin{solution}
	\\a)Give the exact formula...
	\\first for loop executes 2n times,
	\\second for loop executes $(n+1)^2$ times,
	\\thus,\\
	\\$\sum_{n=1}^{2n} (n+1)^{2}$\\
	\\=$\frac{1}{3}(8n^{3}+18n^{2}+13n)$ = exact formula
    \\\\
    \\b)Give the asymptotic value....
    \\because the exact formula is $\frac{1}{3}(8n^{3}+18n^{2}+13n)$
    \\the asymptotic value of the number of "whatsup" is
    \\$O(n^{3})$
\end{solution}

%%%%%%%%%%%%%%%%%%%%%%%%%%%%

\pagebreak

\begin{problem}
	Let $f(n)$ be a function on natural numbers defined
recursively by $f(0) = 1$, $f(1) = 2.3$, $f(n+2) = 3f(n) + f(n+1)$.
Prove by induction that
$(9/4)^n \le f(n) \le (7/3)^n$.
\end{problem}

%-----------------------------

\begin{solution}
	\\Testing base cases:
	\\$n = 0$ , $(9/4)^0\leq f(0) \leq (7/3)^0 = 1 \leq 1 \leq 1$ = True
	\\$n = 1$ , $(9/4)^1 \leq f(1) \leq (7/3)^1 = 2.25 \leq 2.3 \leq 2.333$ = True
	\\
	\\Notice a pattern for f(n),
	\\f(0) = 1
	\\f(1) = 2.3
	\\f(2) = f(0+2) = 3f(0) + f(0+1) = 3(1) + (2.3) = 5.3
	\\f(3) = f(1+2) = 3f(1) + f(1+1) = 3(2.3) + 5.3 = 12.2
	\\
	\\Proving $f(n) = 2.3^n$,
	\\Testing base case;$f(0) = 2.3^0 = 1$ = True\\
	\\Induction:\\$f(n+1) = 2.3^{n+1}$
	\\           $f(1)  = 2.3^1$
	\\           $2.3   = 2.3$ true \\
	\\ So, $f(n)=2.3^n$
	\\ thus, $2.25^n \leq 2.3^n \leq 2.333^n$
	\\ $=(9/4)^n \leq 2.3^n \leq (7/3)^n$
	\\ $=(9/4)^n \le f(n) \le (7/3)^n$
\end{solution}

%%%%%%%%%%%%%%%%%%%%%%%%%%%

\begin{problem}
 	Individual assignment, do not need to complete problem 3
\end{problem}

%----------------------------

\begin{solution}
	Individual assignment, do not need to complete problem 3
\end{solution}

\end{document}
