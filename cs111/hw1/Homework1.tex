% <--- percent sign starts a comment line in LaTeX

%----------------------------------------------------------
% This is a sample assignment .tex file. Put your name,
% assignment number and the due date below, as shown.
% Before you typeset your own assignment try to preview
% and print this one as follows:
%   1. Save this in a file, say hw.tex
%   2. do "latex hw"
%	3. latex will produce a pdf file that should be named hw.pdf.
%	4. Use any pdf viewer to view the document
% ------------------------------------------------------------

\documentclass[11pt]{article}

\usepackage{fullpage,graphicx,latexsym,picinpar,amsbsy,amsmath,amsfonts}

           

%%%%%%%%%%%%%%%%%%%%%%%%%%%%%%%%%%%%%%%%%%%%%%%%%%%%%%%%%%%%%%%%%%%%%%%%%%%%%%%%%%%
%%%%%%%%%%%  LETTERS 
%%%%%%%%%%%%%%%%%%%%%%%%%%%%%%%%%%%%%%%%%%%%%%%%%%%%%%%%%%%%%%%%%%%%%%%%%%%%%%%%%%%

\newcommand{\barx}{{\bar x}}
\newcommand{\bary}{{\bar y}}
\newcommand{\barz}{{\bar z}}
\newcommand{\bart}{{\bar t}}

\newcommand{\bfP}{{\bf{P}}}

%%%%%%%%%%%%%%%%%%%%%%%%%%%%%%%%%%%%%%%%%%%%%%%%%%%%%%%%%%%%%%%%%%%%%%%%%%%%%%%%%%%
%%%%%%%%%%%%%%%%%%%%%%%%%%%%%%%%%%%%%%%%%%%%%%%%%%%%%%%%%%%%%%%%%%%%%%%%%%%%%%%%%%%
                                                                                
\newcommand{\parend}[1]{{\left( #1  \right) }}
\newcommand{\spparend}[1]{{\left(\, #1  \,\right) }}
\newcommand{\angled}[1]{{\left\langle #1  \right\rangle }}
\newcommand{\brackd}[1]{{\left[ #1  \right] }}
\newcommand{\spbrackd}[1]{{\left[\, #1  \,\right] }}
\newcommand{\braced}[1]{{\left\{ #1  \right\} }}
\newcommand{\leftbraced}[1]{{\left\{ #1  \right. }}
\newcommand{\floor}[1]{{\left\lfloor #1\right\rfloor}}
\newcommand{\ceiling}[1]{{\left\lceil #1\right\rceil}}
\newcommand{\barred}[1]{{\left|#1\right|}}
\newcommand{\doublebarred}[1]{{\left|\left|#1\right|\right|}}
\newcommand{\spaced}[1]{{\, #1\, }}
\newcommand{\suchthat}{{\spaced{|}}}
\newcommand{\numof}{{\sharp}}
\newcommand{\assign}{{\,\leftarrow\,}}
\newcommand{\myaccept}{{\mbox{\tiny accept}}}
\newcommand{\myreject}{{\mbox{\tiny reject}}}
\newcommand{\blanksymbol}{{\sqcup}}
                                                                                                                         
\newcommand{\veps}{{\varepsilon}}
\newcommand{\Sigmastar}{{\Sigma^\ast}}
                           
\newcommand{\half}{\mbox{$\frac{1}{2}$}}    
\newcommand{\threehalfs}{\mbox{$\frac{3}{2}$}}   
\newcommand{\domino}[2]{\left[\frac{#1}{#2}\right]}  

%%%%%%%%%%%% complexity classes

\newcommand{\PP}{\mathbb{P}}
\newcommand{\NP}{\mathbb{NP}}
\newcommand{\PSPACE}{\mathbb{PSPACE}}
\newcommand{\coNP}{\textrm{co}\mathbb{NP}}
\newcommand{\DLOG}{\mathbb{L}}
\newcommand{\NLOG}{\mathbb{NL}}
\newcommand{\NL}{\mathbb{NL}}

%%%%%%%%%%% decision problems

\newcommand{\PCP}{\sc{PCP}}
\newcommand{\Path}{\sc{Path}}
\newcommand{\GenGeo}{\sc{Generalized Geography}}

\newcommand{\malytm}{{\mbox{\tiny TM}}}
\newcommand{\malycfg}{{\mbox{\tiny CFG}}}
\newcommand{\Atm}{\mbox{\rm A}_\malytm}
\newcommand{\complAtm}{{\overline{\mbox{\rm A}}}_\malytm}
\newcommand{\AllCFG}{{\mbox{\sc All}}_\malycfg}
\newcommand{\complAllCFG}{{\overline{\mbox{\sc All}}}_\malycfg}
\newcommand{\complL}{{\bar L}}
\newcommand{\TQBF}{\mbox{\sc TQBF}}
\newcommand{\SAT}{\mbox{\sc SAT}}

%%%%%%%%%%%%%%%%%%%%%%%%%%%%%%%%%%%%%%%%%%%%%%%%%%%%%%%%%%%%%%%%%%%%%%%%%%%%%%%%%%%
%%%%%%%%%%%%%%% for homeworks
%%%%%%%%%%%%%%%%%%%%%%%%%%%%%%%%%%%%%%%%%%%%%%%%%%%%%%%%%%%%%%%%%%%%%%%%%%%%%%%%%%%

\newcommand{\student}[2]{%
{\noindent\Large{ \emph{#1} SID {#2} } \hfill} \vskip 0.1in}

\newcommand{\assignment}[1]{\medskip\centerline{\large\bf CS 111 ASSIGNMENT {#1}}}

\newcommand{\duedate}[1]{{\centerline{due {#1}\medskip}}}     

\newcounter{problemnumber}                                                                                 

\newenvironment{problem}{{\vskip 0.1in \noindent
              \bf Problem~\addtocounter{problemnumber}{1}\arabic{problemnumber}:}}{}

\newcounter{solutionnumber}

\newenvironment{solution}{{\vskip 0.1in \noindent
             \bf Solution~\addtocounter{solutionnumber}{1}\arabic{solutionnumber}:}}
				{\ \newline\smallskip\lineacross\smallskip}

\newcommand{\lineacross}{\noindent\mbox{}\hrulefill\mbox{}}

\newcommand{\decproblem}[3]{%
\medskip
\noindent
\begin{list}{\hfill}{\setlength{\labelsep}{0in}
                       \setlength{\topsep}{0in}
                       \setlength{\partopsep}{0in}
                       \setlength{\leftmargin}{0in}
                       \setlength{\listparindent}{0in}
                       \setlength{\labelwidth}{0.5in}
                       \setlength{\itemindent}{0in}
                       \setlength{\itemsep}{0in}
                     }
\item{{{\sc{#1}}:}}
                \begin{list}{\hfill}{\setlength{\labelsep}{0.1in}
                       \setlength{\topsep}{0in}
                       \setlength{\partopsep}{0in}
                       \setlength{\leftmargin}{0.5in}
                       \setlength{\labelwidth}{0.5in}
                       \setlength{\listparindent}{0in}
                       \setlength{\itemindent}{0in}
                       \setlength{\itemsep}{0in}
                       }
                \item{{\em Instance:\ }}{#2}
                \item{{\em Query:\ }}{#3}
                \end{list}
\end{list}
\medskip
}

%%%%%%%%%%%%%%%%%%%%%%%%%%%%%%%%%%%%%%%%%%%%%%%%%%%%%%%%%%%%%%%%%%%%%%%%%%%%%%%%%%%
%%%%%%%%%%%%% for quizzes
%%%%%%%%%%%%%%%%%%%%%%%%%%%%%%%%%%%%%%%%%%%%%%%%%%%%%%%%%%%%%%%%%%%%%%%%%%%%%%%%%%%

\newcommand{\quizheader}{ {\large NAME: \hskip 3in SID:\hfill}
                                \newline\lineacross \medskip }


%%%%%%%%%%%%%%%%%%%%%%%%%%%%%%%%%%%%%%%%%%%%%%%%%%%%%%%%%%%%%%%%%%%%%%%%%%%%%%%%%%%
%%%%%%%%%%%%% for final
%%%%%%%%%%%%%%%%%%%%%%%%%%%%%%%%%%%%%%%%%%%%%%%%%%%%%%%%%%%%%%%%%%%%%%%%%%%%%%%%%%%

\newcommand{\namespace}{\noindent{\Large NAME: \hfill SID:\hskip 1.5in\ }\\\medskip\noindent\mbox{}\hrulefill\mbox{}}



\begin{document}

% v -- YOUR NAME and SID in the braces
\student{ Chris Wong}{ 860 923 521}
% v -- ASSIGNMENT NUMBER in the braces
\assignment{ 1 }
% v-- DUE DATE in the braces
\duedate{Tuesday, April 12 }

\medskip

%%%%%%%%%%%%%%%%%%%%%%%%%%%%%%%%%%%%%%%%%%%%%%%%%%%%%%%%%%%%%%%%%%%%%%%%%%

\lineacross

%%%%%%%%%%%%%%%%%%%%%%%%%%%%

\begin{problem}
	(a)
Give the exact formula (as a function of $n$) for the number of
times ``whatsup" is printed by Algorithm~\textsc{WhatsUp} below.
First express it as a summation, and then give a closed-form expression.

\noindent
(b)
Give the asymptotic value of the
number of ``whatsup"s (using the $\Theta$-notation.)

\begin{tabbing}
aa \= aa \= aa \= aa \= aa \= aa \= \kill
\textbf{Algorithm} \textsc{WhatsUp} $(n: \mbox{\bf integer})$ \\
      \> \textbf{for} $i \leftarrow 1$ \textbf{to} $2n$
                         \textbf{do} \\
      \> \> \textbf{for} $j \leftarrow 1$ \textbf{to} $(i+1)^2$ \textbf{do} \\
      \> \> \> print(``whatsup")
\end{tabbing}
\end{problem}

%---------------------------

\begin{solution}
	\\a)Give the exact formula...
	\\first for loop executes 2n times,
	\\second for loop executes $(n+1)^2$ times,
	\\thus,\\
	\\$\sum_{n=1}^{2n} (n+1)^{2}$\\
	\\=$\frac{1}{3}(8n^{3}+18n^{2}+13n)$ = exact formula
    \\\\
    \\b)Give the asymptotic value....
    \\because the exact formula is $\frac{1}{3}(8n^{3}+18n^{2}+13n)$
    \\the asymptotic value of the number of "whatsup" is
    \\$O(n^{3})$
\end{solution}

%%%%%%%%%%%%%%%%%%%%%%%%%%%%

\pagebreak

\begin{problem}
	Let $f(n)$ be a function on natural numbers defined
recursively by $f(0) = 1$, $f(1) = 2.3$, $f(n+2) = 3f(n) + f(n+1)$.
Prove by induction that
$(9/4)^n \le f(n) \le (7/3)^n$.
\end{problem}

%-----------------------------

\begin{solution}
	\\Testing base cases:
	\\$n = 0$ , $(9/4)^0\leq f(0) \leq (7/3)^0 = 1 \leq 1 \leq 1$ = True
	\\$n = 1$ , $(9/4)^1 \leq f(1) \leq (7/3)^1 = 2.25 \leq 2.3 \leq 2.333$ = True
	\\
	\\Notice a pattern for f(n),
	\\f(0) = 1
	\\f(1) = 2.3
	\\f(2) = f(0+2) = 3f(0) + f(0+1) = 3(1) + (2.3) = 5.3
	\\f(3) = f(1+2) = 3f(1) + f(1+1) = 3(2.3) + 5.3 = 12.2
	\\
	\\Proving $f(n) = 2.3^n$,
	\\Testing base case;$f(0) = 2.3^0 = 1$ = True\\
	\\Induction:\\$f(n+1) = 2.3^{n+1}$
	\\           $f(1)  = 2.3^1$
	\\           $2.3   = 2.3$ true \\
	\\ So, $f(n)=2.3^n$
	\\ thus, $2.25^n \leq 2.3^n \leq 2.333^n$
	\\ $=(9/4)^n \leq 2.3^n \leq (7/3)^n$
	\\ $=(9/4)^n \le f(n) \le (7/3)^n$
\end{solution}

%%%%%%%%%%%%%%%%%%%%%%%%%%%

\begin{problem}
 	Individual assignment, do not need to complete problem 3
\end{problem}

%----------------------------

\begin{solution}
	Individual assignment, do not need to complete problem 3
\end{solution}

\end{document}
