% <--- percent sign starts a comment line in LaTeX

%----------------------------------------------------------
% This is a sample assignment .tex file. Put your name,
% assignment number and the due date below, as shown.
% Before you typeset your own assignment try to preview
% and print this one as follows:
%   1. Save this in a file, say hw.tex
%   2. do "latex hw"
%	3. latex will produce a pdf file that should be named hw.pdf.
%	4. Use any pdf viewer to view the document
% ------------------------------------------------------------

\documentclass[11pt]{article}

\usepackage{fullpage,graphicx,latexsym,picinpar,amsbsy,amsmath,amsfonts}

\input{macros.tex}

\begin{document}

% v -- YOUR NAME and SID in the braces
\student{ Chris Wong}{860-923-521}
% v -- YOUR NAME and SID in the braces
\assignment{5}
% v-- DUE DATE in the braces
\duedate{June 3}

\medskip

%%%%%%%%%%%%%%%%%%%%%%%%%%%%%%%%%%%%%%%%%%%%%%%%%%%%%%%%%%%%%%%%%%%%%%%%%%

\lineacross

%%%%%%%%%%%%%%%%%%%%%%%%%%%%

\begin{problem}
	You are given two bipartite graphs $G$ and $H$ below. For each
graph determine whether it has a perfect matching.
Justify your answer, either by
listing the edges that are in the matching or using
Hall's Theorem to show that the graph does not have a
perfect matching.

\begin{center}
\includegraphics[width = 2in]{bipartite_graphG_hw5.pdf}
\includegraphics[width = 2in]{bipartite_graphH_hw5.pdf}
\end{center}
\end{problem}

%---------------------------

\begin{solution}
	\\Graph G: Impossible, cannot have perfect matching;
    \\Note,  4 can only match to 5 or 9
    \\Note,  3 can only match to 5 or 9
    \\Note,  1 can only match to 5 or 9
    \\SO, 4, and 3, and 1 can only match to 5 or 9, we cannot have perfect match with this.
    \\Thus, graph G does not have perfect matching
    \\
    \\
	\\Graph H: Yes, it has perfect matching;
	\\here is a perfect set:
	\\4 matches with 8,
	\\3 matches with 9,
	\\2 matches with 6,
	\\1 matches with 5,
	\\0 matches with 7
	\\Thus, graph h has perfect matching
\end{solution}

%%%%%%%%%%%%%%%%%%%%%%%%%%%%



\end{document}
