
\documentclass{article}

\usepackage{fullpage,latexsym,picinpar,amsmath,amsfonts,graphicx}

\input{macros.tex}

\begin{document}

\centerline{\large \bf CS/MATH111 ASSIGNMENT 2 (revised April 18)}
\centerline{due Tuesday, April 26 (by midnight)}

\vskip 0.2in
\noindent{\bf Individual assignment:} Problems 1 and 2.

\noindent{\bf Group assignment:} Problems 1,2 and 3.

\vskip 0.2in

%%%%%%%%%%%%%%%%%%%%%%%%%%%%

\begin{problem}
Let $\naturals_k = \braced{1,2,...,k}$ be the set of natural numbers between
$1$ and $k$, where $k$ is some natural number.
For a natural number $x$, by $F(x)$ we denote the set of its prime
factors.
For example, $F(24) = \braced{2,3}$, $F(1100) = \braced{2,5,11}$, etc.
Note that $1$ does not have any prime factors, so $F(1) = \emptyset$.

\medskip
\noindent
(a) We define relation $\bowtie$ on $\naturals_k$ as follows:
$x \bowtie y$ if and only if $F(x) = F(y)$.
For example $6 \bowtie 24$, because $F(6) = \braced{2,3} = F(24)$.

It is easy to see that $\bowtie$ is an equivalence relation on
$\naturals_k$ (you don't have  to prove it).
List all equivalence classes of $\bowtie$ for $\naturals_{30}$.
Use notation $[x]$ for the equivalence class of $x$.
For example, for $\naturals_{10}$, the equivalence classes
are $[1]= \braced{1}$,
$[2] = \braced{2,4,8}$,
$[3] = \braced{3,9}$,
$[5] = \braced{5}$,
$[6] = \braced{6}$,
$[7] = \braced{7}$,
$[10] = \braced{10}$.

\medskip
\noindent
(b)
Now define relation $\unlhd$ on the equivalence classes of
$\bowtie$:
$[x]\unlhd [y]$ if and only if $F(x) \subseteq F(y)$.
For example, $[100]\unlhd [30]$
because $F(100) = \braced{2,5}\subseteq  \braced{2,3,5} = F(30)$.
(To emphasize, relation $\unlhd$ is between \emph{equivalence classes}
of $\bowtie$ on $\naturals_k$, not between the elements of $\naturals_k$.)

Prove that $\unlhd$ is a partial order.
Also, draw the Hasse diagram of $\unlhd$ for $\naturals_{14}$. For example,
the reverse page shows the Hasse diagram of $\unlhd$ for $\naturals_{10}$.

\medskip
Regarding the drawing in (b), you can draw it by hand and scan, or use any
drawing software to create a pdf document. To see how you can include a pdf file
in a {\LaTeX} document, see the source file for this homework.
\end{problem}

%%%%%%%%%%%%%%%%%%%%%%%%%%%%

\medskip

\begin{problem}
In the RSA, Bob chooses $p =11$, $q = 17$.
He is considering three choices for the public
exponent $e$: $5$, $7$ and $33$,
but he's not sure whether they are correct.

\begin{description}

\item{(a)} Which of these three choices for $e$ are correct?
Justify your answer.

\item{(b)} Let now $e$ be the smallest correct choice.
Determine the value of the secret exponent $d$.

\item{(c)} Suppose Alice wants to send $M = 26$ to
Bob. Determine the ciphertext $C$.

\item{(d)} What computation will Bob perform to
decrypt $C$? Show the result.

\end{description}

In parts (b), (c), (d), you don't need to show the details of the
computation (a calculator may be useful for this),
but you need to explain what steps are required to obtain the result.
\end{problem}

%%%%%%%%%%%%%%%%%%%%%%%%%%%%

\medskip

\begin{problem}
Give complete addition and multiplication tables modulo $10$
for the numbers $0,1,...,9$. (Use the {\LaTeX} template on the next
page.) Use the multiplication table to determine which of
those numbers have inverses modulo $10$ and give those inverses.
\end{problem}

%%%%%%%%%%%%%%%%%%%%%%%%%%%%

\vskip 0.1in
\paragraph{Submission.}
You need to turn in your homework via Ilearn, no later than by midnight on
Tuesday, April 26.




%%%%%%%%%%%%%%%%%%%%%%%%%%%%%%%%%%%%%%%%%%%%%%%%%%%%%%%%%%%%%%%%%%%%%

\vfill
\eject

\noindent
Template for Problem~3:

\vskip 0.2in

\begin{tabular}{|c|c|c|c|c|c|c|c|c|c|c|}
\hline
$+$ & 0 & 1 & 2 & 3 & 4 & 5 & 6 & 7 & 8 & 9
\\ \hline
0   & 0 & 1 & 2 & 3 & 4 & 5 & 6 & 7 & 8 & 9
\\ \hline
1   &   &   &   &   &   &   &   &   &  &
\\ \hline
2   &   &   &   &   &   &   &   &   &   &
\\ \hline
3   &   &   &   &   &   &   &   &   &    &
\\ \hline
4   &   &   &   &   &   &   &   &   &    &
\\ \hline
5   &   &   &   &   &   &   &   &   &    &
\\ \hline
6   &   &   &   &   &   &   &   &   &    &
\\ \hline
7   &   &   &   &   &   &   &   &   &   &
\\ \hline
8   &   &   &   &   &   &   &   &   &   &
\\ \hline
9   &   &   &   &   &   &   &   &   &   &
\\ \hline
\end{tabular}

\vskip 1in
\noindent
\hrule
\vskip 1in

Hasse Diagram for $\naturals_{10}$:

\vskip 0.2in

\begin{center}
\includegraphics[width=3in]{hasse_hw2.pdf}
\end{center}

\vfill




\end{document}

